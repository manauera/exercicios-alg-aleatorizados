\documentclass{article}

\renewcommand{\thesection}{}
\renewcommand{\thesubsection}{\arabic{section}.\arabic{subsection}}
\makeatletter
\def\@seccntformat#1{\csname #1ignore\expandafter\endcsname\csname the#1\endcsname\quad}
\let\sectionignore\@gobbletwo
\let\latex@numberline\numberline
\def\numberline#1{\if\relax#1\relax\else\latex@numberline{#1}\fi}
\makeatother



\usepackage[utf8]{inputenc}

\title{Lista de Exercícios 5}
\author{Gustavo Higuchi}
\date{\today}

\usepackage{natbib}
\usepackage{graphicx}
\usepackage{amssymb}
\usepackage{amsthm}
\usepackage{amsmath}
\usepackage{color}   %May be necessary if you want to color links
\usepackage[portuguese, ruled, linesnumbered]{algorithm2e}
\usepackage[normalem]{ulem}

\usepackage{mathtools}
\DeclarePairedDelimiter\ceil{\lceil}{\rceil}
\DeclarePairedDelimiter\floor{\lfloor}{\rfloor}

% usado para linkar cada section na tabela de conteúdo com a respectiva
% página no documento
\usepackage{hyperref}
\hypersetup{
    colorlinks,
    citecolor=black,
    filecolor=black,
    linkcolor=black,
    urlcolor=black,
    linktoc=all
}

%o começo do documento
\begin{document}

% compila o título
\maketitle

% compila a tabela de conteúdos
\tableofcontents
\newpage


\chapter{}
\section{Exercício 1}
\begin{proof}
    Assumindo que temos
    \begin{equation}
    \notag
        \begin{split}
            X &= \{1,2,3,4,5,6\}\\
            X^2 &= \{1,2,3,4,5,6\}\\
            &\\
            Y &=\{2,3,4,5,6,7,8,9,10,11,12\}
        \end{split}
    \end{equation}

    Então, pela \textit{linearidade da esperança}, temos:
    \begin{equation}
    \notag
        \begin{split}
            E[X+X^2]& = E[X] + E[X^2]\\
            & = 3,5 + 3,5\\
            & = 7
        \end{split}
    \end{equation}

    E se calcularmos a esperança de Y isoladamente, temos
    \begin{equation}
    \notag
        \begin{split}
            E[Y] &= \sum\limits_{i} i.Pr(Y = i)\\
            &= 2.\dfrac{1}{11} + 3.\dfrac{1}{11} + 4.\dfrac{1}{11} + \ldots + 12.\dfrac{1}{11}\\
            &= \dfrac{1}{11}.(2+3+4+5+6+7+8+9+10+11+12)\\
            & = \dfrac{1}{11}. 77 = 7
        \end{split}
    \end{equation}
\end{proof}
\section{Exercício 2}
\subsection*{(a)}
Levando em consideração que se eu tenho $X_1 = 2$, a chance de $X = 1$ é $Pr(X=1) = 0$
porque se já tenho um dado que vale 2, e quero o maximo de dois valores, a chance de 
ter um resultado $max(X_1,X_2) = 1$ é zero. Então considero apenas os valores que
são possíveis. Ficamos com uma conta parecida com o seguinte:
\begin{equation}
\notag
    \begin{split}
        E[max(X_1,X_2)] &= \sum\limits_{x_1}\sum\limits_{x_2}max(x_1, x_2).\dfrac{1}{k}.\dfrac{1}{k}\\
        &= \dfrac{1}{k^2}\sum\limits_{x_1}\sum\limits_{x_1 \geq x_2}x_1 + \sum\limits_{x_2 > x_1}x_2\\
        &\\
        E[min(X_1,X_2)] &= \sum\limits_{x_1}\sum\limits_{x_2}min(x_1, x_2).\dfrac{1}{k}.\dfrac{1}{k}\\
        &= \dfrac{1}{k^2}\sum\limits_{x_1}\sum\limits_{x_1 < x_2}x_1 + \sum\limits_{x_2 \leq x_1}x_2
    \end{split}
\end{equation}
\subsection*{(b)}

\begin{equation}
\notag
    \begin{split}
        E[max(X_1,X_2)] + E[min(X_1,X_2)] &= \left(\dfrac{1}{k^2}\sum\limits_{x_1}\sum\limits_{x_1 \geq x_2}x_1 + \sum\limits_{x_2 > x_1}x_2\right) + \left(\dfrac{1}{k^2}\sum\limits_{x_1}\sum\limits_{x_1 < x_2}x_1 + \sum\limits_{x_2 \leq x_1}x_2\right)\\
        &=  \dfrac{1}{k^2}\sum\limits_{x_1}\left(\sum\limits_{x_1 \geq x_2}x_1+ \sum\limits_{x_1 < x_2}x_1 + \sum\limits_{x_2 \leq x_1}x_2 + \sum\limits_{x_2 > x_1}x_2 \right)\\
        &= \dfrac{1}{k^2}\sum\limits_{x_1}\left(\sum\limits_{x_2 }x_1 + \sum\limits_{x_2}x_2\right)\\
        &= \dfrac{1}{k^2}\sum\limits_{x_1}\sum\limits_{x_2}x_1 + x_2\\
        &= E[X_1] + E[X_2]
    \end{split}
\end{equation}
\subsection*{(c)}
Pela \textit{linearidade da esperança}, temos
\begin{equation}
\notag
    \begin{split}
        E[max(X_1,X_2)] + E[min(X_1,X_2)] &= E[max(X_1,X_2) + min(X_1,X_2)]\\
        &= E[X_1 + X_2]\\
        &= E[X_1] + E[X_2]
    \end{split}
\end{equation}

\end{document}
