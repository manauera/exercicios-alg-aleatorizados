\documentclass{article}

\renewcommand{\thesection}{}
\renewcommand{\thesubsection}{\arabic{section}.\arabic{subsection}}
\makeatletter
\def\@seccntformat#1{\csname #1ignore\expandafter\endcsname\csname the#1\endcsname\quad}
\let\sectionignore\@gobbletwo
\let\latex@numberline\numberline
\def\numberline#1{\if\relax#1\relax\else\latex@numberline{#1}\fi}
\makeatother



\usepackage[utf8]{inputenc}

\title{Lista de algoritmos aleatorizados}
\author{Gustavo Higuchi}
\date{\today}

\usepackage{natbib}
\usepackage{graphicx}
\usepackage{amssymb}
\usepackage{amsthm}
\usepackage{amsmath}
\usepackage{color}   %May be necessary if you want to color links
\usepackage[portuguese, ruled, linesnumbered]{algorithm2e}
\usepackage[normalem]{ulem}

\usepackage{mathtools}
\DeclarePairedDelimiter\ceil{\lceil}{\rceil}
\DeclarePairedDelimiter\floor{\lfloor}{\rfloor}

% usado para linkar cada section na tabela de conteúdo com a respectiva
% página no documento
\usepackage{hyperref}
\hypersetup{
    colorlinks,
    citecolor=black,
    filecolor=black,
    linkcolor=black,
    urlcolor=black,
    linktoc=all
}

%o começo do documento
\begin{document}

% compila o título
\maketitle

% compila a tabela de conteúdos
\tableofcontents
\newpage


\chapter{}
\section{Exercício 1}
\subsection*{(a)}
\begin{proof}
	Aplicando \emph{Markov}, temos
	\begin{equation}
		\begin{split}
			Pr(X \geq (1 + \delta)U) &= Pr(e^{tX} \geq e^{(1+\delta)U})\\
			&\leq \dfrac{E[e^{tX}]}{e^{t(1+\delta)U}}\\
			&\leq \dfrac{e^{(e^t - 1)U}}{e^{t(1+\delta)U}}\\
			\text{então, para um }\delta > 0 & \text{ podemos setar }t = ln(1+\delta)\\
			Pr(X \geq (1 + \delta)U) &\leq \left(\dfrac{e^{\delta}}{{(1+\delta)^{(1 + \delta)}}}\right)^{U}
		\end{split}
	\end{equation}
\end{proof}

\subsection*{(b)}
\begin{proof}
	Para um $0 < \delta \leq 1$, vamos mostrar que
	\begin{equation}
	\notag
		Pr( X \geq (1 + \delta)U) \leq e^{\dfrac{-U\delta^2}{3}}
	\end{equation}
	Pela prova anterior
	\begin{equation}
		\notag
		\left(\dfrac{e^{\delta}}{{(1+\delta)^{(1 + \delta)}}}\right)^{U} \leq e^{\dfrac{-U\delta^2}{3}}
	\end{equation}
	Aplicando a \emph{raiz U}, teremos
	\begin{equation}
		\notag
		\dfrac{e^{\delta}}{(1+\delta)^{(1 + \delta)}} \leq e^{-\delta^2/3}
	\end{equation}

	e tirando o logaritmo de cada lado ficamos com
	\begin{equation}
		\notag
			f(\delta) = \delta - (1 + \delta)ln(1+\delta) + \dfrac{\delta^2}{3} \leq 0\\
	\end{equation}

	Tirando a derivada primeira e segunda, ficamos com seguinte
	\begin{equation}
		\begin{split}
			f'(\delta) &= 1 - \dfrac{1+\delta}{1+\delta} - ln(1+\delta) + \dfrac{2\delta}{3}\\
			&= - ln(1+\delta) + \dfrac{2\delta}{3}\\
			f''(\delta)&= -\dfrac{1}{1+\delta} + \dfrac{2}{3}
		\end{split}
	\end{equation}

	Como temos uma segunda derivada menor que zero ($f''(\delta) < 0$) para um $0 \leq \delta <
	 1/2$, e maior que zero ($f''(\delta) > 0$) para um $\delta > 1/2$. Portanto, a primeira 
	 derivada $f'(\delta)$ aumenta no interval $[0,1]$. Então, uma vez que $f'(0) = 0$ e 
	 $f'(1) < 0$, podemos concluir que $f'(\delta) \leq 0$ no intervalo $[0,1]$.\\

	 Assim, como $f(0) = 0$, temos uma função que começa em zero e decresce no intervalo $[0,1]$, provando o item (b).
\end{proof}

\section{Exercício 2}
\subsection*{(a)}
\begin{proof}
	Usando \emph{desigualdade de Markov}, para qualquer $t < 0$, temos
	\begin{equation}
		\notag
		\begin{split}
			Pr(X \leq (1-\delta)\mu) & = Pr(e^{tX} \geq e^{t(1-\delta)\mu})\\
			&\leq \dfrac{E[e^{tX}]}{e^{t(1-\delta)\mu}}\\
			&\leq \dfrac{e^{(e^t-1)\mu}}{e^{t(1-\delta)\mu}}
		\end{split}
	\end{equation}

	Para um $0 < \delta < 1$, podemos setar $t = ln(1-\delta)$ para obter
	\begin{equation}
		\begin{split}
			Pr(X \leq (1-\delta)\mu) &\leq \dfrac{e^{(e^{ln(1-\delta)}-1)\mu}}{e^{ln(1 - \delta)(1-\delta)\mu}}\\
			&\leq \dfrac{e^{(1-\delta - 1)\mu}}{(1-\delta)^{(1-\delta)\mu}}\\
			&\leq \left(\dfrac{e^{-\delta}}{(1-\delta)^{(1-\delta)}}\right)^{\mu}
		\end{split}
	\end{equation}
\end{proof}

\subsection*{(b)}
\begin{proof}
	Para um $0 < \delta < 1$, mostrar que
	\begin{equation}
		\notag
		\begin{split}
			\left(\dfrac{e^{-\delta}}{(1-\delta)^{(1-\delta)}} \right)^{\mu} &\leq e^{-\mu\delta^2/2}\\
			\dfrac{e^{-\delta}}{(1-\delta)^{(1-\delta)}} \leq e^{-\delta^2/2}
		\end{split}
	\end{equation}

	Aplicando logaritmo dos dois lados, temos para um $0 < \delta < 1$
	\begin{equation}
		\notag
		f(\delta) = -\delta - (1-\delta)ln(1-\delta) + \dfrac{\delta^2}{2} \leq 0
	\end{equation}

	Tirando as derivadas, ficamos com
	\begin{equation}
		\begin{split}
			f'(\delta) &= ln(1-\delta) + \delta\\
			f''(\delta) &= -\dfrac{1}{1-\delta} + 1
		\end{split}
	\end{equation}

	Uma vez que a segunda derivada é menor que zero ($f''(\delta)<0$) no intervalo (0,1) e
	como temos $f'(0) = 0$ e $f'(\delta)\leq 0$ no intervalo [0,1). Então podemos concluir que
	$f(\delta) \leq 0$ para um $0 < \delta < 1$, provando item (b).
\end{proof}
\end{document}
