\documentclass{article}

\renewcommand{\thesection}{}
\renewcommand{\thesubsection}{\arabic{section}.\arabic{subsection}}
\makeatletter
\def\@seccntformat#1{\csname #1ignore\expandafter\endcsname\csname the#1\endcsname\quad}
\let\sectionignore\@gobbletwo
\let\latex@numberline\numberline
\def\numberline#1{\if\relax#1\relax\else\latex@numberline{#1}\fi}
\makeatother



\usepackage[utf8]{inputenc}

\title{Lista de Exercícios 12}
\author{Gustavo Higuchi}
\date{\today}

\usepackage{natbib}
\usepackage{graphicx}
\usepackage{amssymb}
\usepackage{amsthm}
\usepackage{amsmath}
\usepackage{color}   %May be necessary if you want to color links
\usepackage[portuguese, ruled, linesnumbered]{algorithm2e}
\usepackage[normalem]{ulem}

\usepackage{mathtools}
\DeclarePairedDelimiter\ceil{\lceil}{\rceil}
\DeclarePairedDelimiter\floor{\lfloor}{\rfloor}

% usado para linkar cada section na tabela de conteúdo com a respectiva
% página no documento
\usepackage{hyperref}
\hypersetup{
    colorlinks,
    citecolor=black,
    filecolor=black,
    linkcolor=black,
    urlcolor=black,
    linktoc=all
}

%o começo do documento
\begin{document}

% compila o título
\maketitle

% compila a tabela de conteúdos
\tableofcontents
\newpage


\chapter{}
\section{Exercício 1}
\hspace*{30pt}No caso do lema 1.5 do livro, temos que algum elemento
da matriz tem que ser diferente de zero, então não há perda de generalidade 
se definir que o elemento é o $D_{11}$, do mesmo jeito que pode-se definir 
qualquer outro elemento, geralmente a escolha é feita para simplificação 
da prova.

\section{Exercício 2}
\hspace*{30pt}A probabilidade condicional se refere à probabilidade
de um evento A ocorrer, dado que um outro evento B já tenha ocorrido.

Caso, sejam independentes, não muda a probabilidade de A acontecer, mas
se forem dependentes a probabilidade de A ocorrer pode ser maior ou menor
dependendo da situação.

Para casos em que queremos que A ocorra, a relação de tempo físico,
seria que se B ocorreu, A eventualmente irá ocorrer. 

\section{Exercício 3}
\textit{Suppose that we roll ten standard six-sided dice. What is the probability 
that their sum will be divisible by 6, assuming that the rolls are independent? }

\hspace*{30pt} Se jogarmos \textbf{9 dados} e seja o resultado da soma parcial $x$,
também considerar que o resultado do último dado seja $n \in \{1,2,3,4,5,6\}$, então
para que a soma $n + x$ seja divisível por 6, o último dado tem que ser $n = 6$,
independente do valor de $x$. Então temos que a probabilidade que estamos procurando,
em todas as possíveis somas, é:
\begin{equation}
\notag
    Pr(n=6) = \dfrac{1}{6}
\end{equation}

\section{Exercício 4}
\subsection*{(a)}
\textit{Consider the set {1, ... , n}. We generate a subset X of this set as fol-
lows: a fair coin is flipped independently for each element of the set: if the coin lands
heads then the element is added to X, and otherwise it is not. Argue that the resulting
set X is equally likely to be anyone of the 2" possible subsets.}\\

\< não soube responder a tempo de entregar \>

\subsection*{(b)}
\textit{Suppose that two sets X and Yare chosen independently and uniformly at ran-
dom from all the 211 subsets of {l, ... ,n}. Determine Pr(X ~ Y) and Pr(X U Y =
11 .... , n D. (Hint: Use the part (a) of this problem.)}\\

\< não soube responder a tempo de entregar \>

\end{document}
