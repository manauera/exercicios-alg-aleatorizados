\documentclass{article}

\renewcommand{\thesection}{}
\renewcommand{\thesubsection}{\arabic{section}.\arabic{subsection}}
\makeatletter
\def\@seccntformat#1{\csname #1ignore\expandafter\endcsname\csname the#1\endcsname\quad}
\let\sectionignore\@gobbletwo
\let\latex@numberline\numberline
\def\numberline#1{\if\relax#1\relax\else\latex@numberline{#1}\fi}
\makeatother



\usepackage[utf8]{inputenc}

\title{Lista de Exercícios 12}
\author{Gustavo Higuchi}
\date{\today}

\usepackage{natbib}
\usepackage{graphicx}
\usepackage{amssymb}
\usepackage{amsthm}
\usepackage{amsmath}
\usepackage{color}   %May be necessary if you want to color links
\usepackage[portuguese, ruled, linesnumbered]{algorithm2e}
\usepackage[normalem]{ulem}

\usepackage{mathtools}
\DeclarePairedDelimiter\ceil{\lceil}{\rceil}
\DeclarePairedDelimiter\floor{\lfloor}{\rfloor}

% usado para linkar cada section na tabela de conteúdo com a respectiva
% página no documento
\usepackage{hyperref}
\hypersetup{
    colorlinks,
    citecolor=black,
    filecolor=black,
    linkcolor=black,
    urlcolor=black,
    linktoc=all
}

%o começo do documento
\begin{document}

% compila o título
\maketitle

% compila a tabela de conteúdos
\tableofcontents
\newpage


\chapter{}
\section{Exercício 1}
Os eventos $\mathcal{A}$ e $\mathcal{B}$ podem ser independentes, se e somente se
\begin{equation}
 	Pr(\mathcal{A} \cap \mathcal{B}) = Pr(\mathcal{A}) . Pr(\mathcal{B})
 \end{equation} 

\section{Exercício 2}
\newtheorem{teo1}{Teorema}
\begin{teo1}
    Demonstre que a Lei das Probabilidades Totais é verdadeira
\end{teo1}
\begin{proof}
    \hfill \break
    Como 
    \begin{equation}
    \notag
        \bigcup_{i=1}^n E_i = \Omega
    \end{equation}
    então
    \begin{equation}
    	B = B \cap \Omega = B \cap \bigcup_{i=1}^{n}E_i = \bigcup_{i=1}^n {B \cap E_i}
    \end{equation}
    Como os conjuntos são disjuntos, podemos afirmar que para $i \neq j, \{B \cap E_i\} \cap \{B \cap E_j\} = emptyset$
    
    Daí temos
    \begin{equation}
        Pr(B) = \sum\limits_{i = 1}^{n}Pr(B \cap E_i)
    \end{equation}

    A partir da Definição 1.3 do livro temos
    \begin{equation}
    	Pr(B) = \sum\limits_{i = 1}^{n}Pr(B | E_i) . Pr(E_i)
    \end{equation}
\end{proof}

\section{Exercício 3}
Dados duas moedas, uma com duas caras e outra normal, temos
\begin{equation}
	\notag
	\begin{split}
		\text{chance de cair cara }H =& \dfrac{3}{4}\\
		\text{chance de cair coroa }T =& \dfrac{1}{4}\\
	\end{split}
\end{equation}
e também
\begin{equation}
	\notag
	\begin{split}
		\text{chance de pegar moeda viciada }MV =& \dfrac{1}{2}\\
		\text{chance de pegar moeda normal }MN =& \dfrac{1}{2}\\
	\end{split}
\end{equation}
Levando em conta que pegamos uma moeda aleatoriamente e o resultado é $cara$,
então a probabilidade de termos o evento $MV$ dado que $H$, temos que a probabilidade
de ter tirado a moeda viciada seja de:

\begin{equation}
	\notag
	\begin{split}
		Pr(MV | H).Pr(H) = & \dfrac{Pr(MV \cap H)}{Pr(H)}\\
        & \dfrac{}
        &\dfrac{1}{2}*\dfrac{3}{4} = \dfrac{3}{8}
	\end{split}
\end{equation}
\end{document}
