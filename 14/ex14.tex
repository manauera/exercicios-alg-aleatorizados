\documentclass{article}

\renewcommand{\thesection}{}
\renewcommand{\thesubsection}{\arabic{section}.\arabic{subsection}}
\makeatletter
\def\@seccntformat#1{\csname #1ignore\expandafter\endcsname\csname the#1\endcsname\quad}
\let\sectionignore\@gobbletwo
\let\latex@numberline\numberline
\def\numberline#1{\if\relax#1\relax\else\latex@numberline{#1}\fi}
\makeatother



\usepackage[utf8]{inputenc}

\title{Lista de algoritmos aleatorizados}
\author{Gustavo Higuchi}
\date{\today}

\usepackage{natbib}
\usepackage{graphicx}
\usepackage{amssymb}
\usepackage{amsthm}
\usepackage{amsmath}
\usepackage{color}   %May be necessary if you want to color links
\usepackage[portuguese, ruled, linesnumbered]{algorithm2e}
\usepackage[normalem]{ulem}

\usepackage{mathtools}
\DeclarePairedDelimiter\ceil{\lceil}{\rceil}
\DeclarePairedDelimiter\floor{\lfloor}{\rfloor}

% usado para linkar cada section na tabela de conteúdo com a respectiva
% página no documento
\usepackage{hyperref}
\hypersetup{
    colorlinks,
    citecolor=black,
    filecolor=black,
    linkcolor=black,
    urlcolor=black,
    linktoc=all
}

%o começo do documento
\begin{document}

% compila o título
\maketitle

% compila a tabela de conteúdos
\tableofcontents
\newpage


\chapter{}
\section{Exercício 1}
\hspace*{15pt} Paradoxo é o oposto do que alguém pensa em ser
verdade. Por isso consideramos o "paradoxo do aniversário" como
sendo um paradoxo. Ao contrário do que se pensa, precisamos de poucas
pessoas para se ter uma chance grande de ter pessoas com a mesma data 
de aniversário.

\section{Exercício 2}
Assumindo que algo seja provável de acontecer tenha mais de 50\% de acontecer.\\

Levando em conta que $m$ é a quantidade de pessoas, $n$ é a quantidade de 
números possíveis e $p$ a probabilidade que queremos, temos\\

\begin{equation}
	\notag
	\begin{split}
		m &\approx \sqrt{-2n \ln p}\\
		&= \sqrt{-2.10^4 \ln(1/2)}\\
		&\approx 118
	\end{split}
\end{equation}

Para que seja provável que exista em uma sala duas pessoas com os últimos 
quatro números de registro iguais, temos que ter 118 pessoas nessa sala.

Para que se tenha duas pessoas com exatamente o mesmo número de registro,
apenas mudamos a quantidade de número disponíveis, ficamos com o seguinte
\begin{equation}
	\notag
	\begin{split}
		m &\approx \sqrt{-2n \ln p}\\
		&= \sqrt{-2.10^9 \ln (1/2)}\\
		&\approx 37.233
	\end{split}
\end{equation}

Ou seja, a partir de $37.233$ pessoas, as chances são mais que 50\% 
de termos duas pessoas com o mesmo número de registro.

Se refizermos essa conta com um número de $11$ dígitos, ficamos com o seguinte

\begin{equation}
\notag
	\begin{split}
		m &\approx \sqrt{-2n \ln p}\\
		&= \sqrt{-2.10^{11} \ln (1/2)}\\
		&\approx 372.329
	\end{split}
\end{equation}

O número de pessoas seria $372.329$, $10$x mais.

\section{Exercício 3}
Para calcular a probabilidade de se ter 3 pessoas com a mesma data de aniversário, é
preciso calcular o complementar das probabilidades \emph{a probabilidade de se ter todas
 datas distintas}, \emph{a probabilidade de se ter um par e as outras datas distintas}, 
 \emph{a probabilidade de se ter dois pares e outras datas distintas}, etc.

Então, podemos resumir a probabilidade de se ter ao menos 3 pessoas com datas de 
aniversário iguais em um grupo de 100 alunos como sendo definido por:

\begin{equation}
	\notag
	Pr(\text{ao menos 3 pessoas com o mesmo aniversário}) = 1 - \sum\limits_{i=0}^{n/2} Pr(E_{i})
\end{equation}
onde $E_i$ é o evento que $i$ pares de pessoas compartilhem o mesmo aniversário. Ou seja,
$E_0$ é o evento onde todos os aniversários são distintos, $E_1$ onde há um par que 
compartilhem o mesmo aniversário e o resto sejam distintos, etc.


\end{document}
