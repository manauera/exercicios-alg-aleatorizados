\documentclass{article}

\renewcommand{\thesection}{}
\renewcommand{\thesubsection}{\arabic{section}.\arabic{subsection}}
\makeatletter
\def\@seccntformat#1{\csname #1ignore\expandafter\endcsname\csname the#1\endcsname\quad}
\let\sectionignore\@gobbletwo
\let\latex@numberline\numberline
\def\numberline#1{\if\relax#1\relax\else\latex@numberline{#1}\fi}
\makeatother



\usepackage[utf8]{inputenc}

\title{Lista de algoritmos aleatorizados}
\author{Gustavo Higuchi}
\date{\today}

\usepackage{natbib}
\usepackage{graphicx}
\usepackage{amssymb}
\usepackage{amsthm}
\usepackage{amsmath}
\usepackage{color}   %May be necessary if you want to color links
\usepackage[portuguese, ruled, linesnumbered]{algorithm2e}
\usepackage[normalem]{ulem}

\usepackage{mathtools}
\DeclarePairedDelimiter\ceil{\lceil}{\rceil}
\DeclarePairedDelimiter\floor{\lfloor}{\rfloor}

% usado para linkar cada section na tabela de conteúdo com a respectiva
% página no documento
\usepackage{hyperref}
\hypersetup{
    colorlinks,
    citecolor=black,
    filecolor=black,
    linkcolor=black,
    urlcolor=black,
    linktoc=all
}

%o começo do documento
\begin{document}

% compila o título
\maketitle

% compila a tabela de conteúdos
\tableofcontents
\newpage


\chapter{}
\section{Exercício 1}
\subsection*{(a)}
\begin{equation}
\notag
	\begin{split}
		M_X(t) = E[e^{tX}] &= \sum\limits_{k=0}^n e^{tk} {n \choose k} p^k(1-p)^{n-k}\\
		&= \sum\limits_{k=0}^n{n \choose k}(pe^t)^k (1-p)^{n-k}\\
		\text{pelo bin. de Newton}&\\
		&= (pe^t + 1-p)^n
	\end{split}
\end{equation}

\subsection*{(b)}
Pelo teorema 4.3 do MU, temos
\begin{equation}
\notag
	\begin{split}
		M_{X+Y}(t) &= M_X(t)M_Y(t)\\
		&= (pe^t + 1-p)^n . (pe^t + 1-p)^m\\
		&= (pe^t + 1-p)^{n+m}
	\end{split}
\end{equation}

\subsection*{(c)}
Podemos concluir que a V.A. $Z = X + Y$, onde $X$ e $Y$ possuem a mesma probabilidade $p$ de 
sucesso e $X$ com $n$ tentativas e $Y$ com $m$ tentativas, i.e. $Z = B(n+m, p)$, é uma \emph{f.g.m.} de $n+m$ tentativas e probabilidade $p$.
\end{document}
