\documentclass{article}

\renewcommand{\thesection}{}
\renewcommand{\thesubsection}{\arabic{section}.\arabic{subsection}}
\makeatletter
\def\@seccntformat#1{\csname #1ignore\expandafter\endcsname\csname the#1\endcsname\quad}
\let\sectionignore\@gobbletwo
\let\latex@numberline\numberline
\def\numberline#1{\if\relax#1\relax\else\latex@numberline{#1}\fi}
\makeatother



\usepackage[utf8]{inputenc}

\title{Lista de Exercícios 8}
\author{Gustavo Higuchi}
\date{\today}

\usepackage{natbib}
\usepackage{graphicx}
\usepackage{amssymb}
\usepackage{amsthm}
\usepackage{amsmath}
\usepackage{color}   %May be necessary if you want to color links
\usepackage[portuguese, ruled, linesnumbered]{algorithm2e}
\usepackage[normalem]{ulem}

\usepackage{mathtools}
\DeclarePairedDelimiter\ceil{\lceil}{\rceil}
\DeclarePairedDelimiter\floor{\lfloor}{\rfloor}

% usado para linkar cada section na tabela de conteúdo com a respectiva
% página no documento
\usepackage{hyperref}
\hypersetup{
    colorlinks,
    citecolor=black,
    filecolor=black,
    linkcolor=black,
    urlcolor=black,
    linktoc=all
}

%o começo do documento
\begin{document}

% compila o título
\maketitle

% compila a tabela de conteúdos
\tableofcontents
\newpage


\chapter{}
\section{Exercício 1}
\hspace*{15pt}Para uma permutação de tamanho $n$, então seja $X_{ij} = 1$ 
se $(i,j)$ é uma inversão, $X_{ij} = 0$ c.c . Então, 

\begin{equation}
\notag
	E[X_{ij}] = Pr(\{i,j\}\text{ ser uma inversão}) = \dfrac{1}{2}
\end{equation}

É fácil de ver que para um par $(i,j)$ que é uma inversão, o par $(j,i)$ não é uma 
inversão.\\

Disso temos\\

\begin{equation}
\notag
\begin{split}
	E[\sum\limits_{i=1}^{n-1}\sum\limits_{j=i+1}^{n}X_{ij}] &= \sum\limits_{i=1}^{n-1}\sum\limits_{j=i+1}^{n}E[X_{ij}] \\
	&= \sum\limits_{i=1}^{n-1}\sum\limits_{j=i+1}^{n} \dfrac{1}{2}\\
	&= \dfrac{1}{2}\sum\limits_{i=1}^{n-1}\sum\limits_{j=i+1}^{n} 1\\
	&= \dfrac{1}{2}\sum\limits_{i=1}^{n-1} (n - (i+1) + 1)\\
	&= \dfrac{1}{2}\sum\limits_{i=1}^{n-1} (n-i)\\
	&= \dfrac{1}{2}\left[\sum\limits_{i=1}^{n-1} n - \sum\limits_{i=1}^{n-1} i \right]\\
	&= \dfrac{1}{2}\left[n(n-1) - \dfrac{n(n-1)}{2} \right]\\
	&= \dfrac{n(n-1)}{4}
\end{split}
\end{equation}

\section{Exercício 2}
\hspace*{15pt}O número esperado de trocas realizadas no k-ésimo elemento, será

\begin{equation}
	\notag
	E[\text{trocas}] = \sum\limits_{i=0}^{k-1} i \dfrac{1}{k} = \dfrac{k-1}{2}
\end{equation}

Então, o número esperado de trocas para todo o insertion sort dado uma permutação
aleatória é a soma de todas as esperanças até $n$, então temos que o número esperado
de trocas realizadas em toda execução do algoritmo para um vetor aleatório é:

\begin{equation}
\notag
	\sum\limits_{i=1}^{n}E[\text{trocas}] = \sum\limits_{i=1}^{n} \dfrac{(i-1)}{2} =\dfrac{n(n-1)}{4}
\end{equation}
\end{document}
