\documentclass{article}

\renewcommand{\thesection}{}
\renewcommand{\thesubsection}{\arabic{section}.\arabic{subsection}}
\makeatletter
\def\@seccntformat#1{\csname #1ignore\expandafter\endcsname\csname the#1\endcsname\quad}
\let\sectionignore\@gobbletwo
\let\latex@numberline\numberline
\def\numberline#1{\if\relax#1\relax\else\latex@numberline{#1}\fi}
\makeatother



\usepackage[utf8]{inputenc}

\title{Lista de Exercícios 12}
\author{Gustavo Higuchi}
\date{\today}

\usepackage{natbib}
\usepackage{graphicx}
\usepackage{amssymb}
\usepackage{amsthm}
\usepackage{amsmath}
\usepackage{color}   %May be necessary if you want to color links
\usepackage[portuguese, ruled, linesnumbered]{algorithm2e}
\usepackage[normalem]{ulem}

\usepackage{mathtools}
\DeclarePairedDelimiter\ceil{\lceil}{\rceil}
\DeclarePairedDelimiter\floor{\lfloor}{\rfloor}

% usado para linkar cada section na tabela de conteúdo com a respectiva
% página no documento
\usepackage{hyperref}
\hypersetup{
    colorlinks,
    citecolor=black,
    filecolor=black,
    linkcolor=black,
    urlcolor=black,
    linktoc=all
}

%o começo do documento
\begin{document}

% compila o título
\maketitle

% compila a tabela de conteúdos
\tableofcontents
\newpage


\chapter{}
\section{Exercício 1}
\newtheorem{teo1}{Teorema}
\begin{teo1}
    Dados eventos $E_1, E_2$, ... temos
    \begin{equation}
    \notag
        Pr\left(\bigcup\limits_{i \geq 1}^{n} E_i\right) \leq \sum\limits_{i \geq 1}^{n}Pr(E_i)
    \end{equation}
\end{teo1}
\begin{proof}
    \hfill \break
    \hspace*{20pt} \textbf{Base} : Para $i = 2$, temos
    \begin{equation}
    \notag
        \begin{split}
            Pr(E_1 \cup E_2) \leq& Pr(E_1) + Pr(E_2)\\
            Pr(E_1) + Pr(E_2) - Pr(E_1 \cap E_2) \leq& Pr(E_1) + Pr(E_2)
        \end{split}
    \end{equation}
    \hspace*{30pt} o que é verdade.\\

    \textbf{Hipótese} : Vamos assumir que a equação
    \begin{equation}
        Pr\left(\bigcup\limits_{i \geq 1}^{n} E_i\right) \leq \sum\limits_{i \geq 1}^{n}Pr(E_i)
    \end{equation}
    \hspace*{30pt} está correta para um $k \leq n$.\\

    \textbf{Passo} : quero provar que para $k = n + 1$ a o hipótese segura

    Assim temos \newline
    \begin{equation}
    \notag
        \begin{split}
            Pr\left(\bigcup\limits_{i \geq 1}^{n+1} E_i \right) \leq& \sum\limits_{i \geq 1}^{n+1}Pr(E_i)\\
            Pr\left(\bigcup\limits_{i \geq 1}^{n} E_i \cup E_{n+1}\right) \leq & \sum\limits_{i \geq 1}^{n}Pr(E_i) + Pr(E_{n+1})\\
            Pr\left(\bigcup\limits_{i \geq 1}^{n} E_i\right) + Pr(E_{n+1}) - Pr\left(\left[\bigcup\limits_{i \geq 1}^{n} E_i\right]\cap E_{n+1}\right)  \leq &Pr(E_{n+1})\\
            \text{pela hipótese de indução, temos}&\\
            Pr(E_{n+1}) - Pr\left(\left[\bigcup\limits_{i \geq 1}^{n} E_i\right]\cap E_{n+1}\right) \leq & Pr(E_{n+1})
        \end{split}
    \end{equation}
    Sendo assim, um número é sempre menor ou igual a ele mesmo, então a hipótese é verdadeira.
\end{proof}

\section{Exercício 2}
\newtheorem{teo2}{Teorema}
\begin{teo2}
    Prove o princípio da Inclusão-Exclusão
    \begin{equation}
    \notag
        \begin{split}
            Pr\left(\bigcup_{i = 1}^{n}E_i\right) =& \sum\limits_{i=1}^{n}Pr(E_i)\\
            & -\sum\limits_{1\leq i<j\leq n}^{n}Pr(E_i \cap E_j)\\
            & +\sum\limits_{1\leq i<j<k\leq n}^{n}Pr(E_i \cap E_j \cap E_k)\\
            & \ldots\\
            & +(-1)^{n-1}Pr\left(\bigcap_{i=1}^{n}E_i\right)
        \end{split}
    \end{equation}
\end{teo2}
\begin{proof}
    \hfill \break
    \textbf{Base} : Para $n = 2$, temos
    \begin{equation}
    \notag
        Pr(E_1 \cup E_2) = Pr(E_1) + Pr(E_2)- Pr(E_1 \cap E_2)
    \end{equation}
    \hspace*{30pt} o que é verdade.\\

    \textbf{Hipótese} : Vamos assumir que a equação seja verdade para um $r \leq n$
    \begin{equation}
    \notag
        \begin{split}
            Pr\left(\bigcup_{i = 1}^{r}E_i\right) =& \sum\limits_{i=1}^{r}Pr(E_i)\\
            & -\sum\limits_{1\leq i<j\leq r}^{r}Pr(E_i \cap E_j)\\
            & +\sum\limits_{1\leq i<j<k\leq r}^{r}Pr(E_i \cap E_j \cap E_k)\\
            & \ldots\\
            & +(-1)^{r-1}Pr(\bigcap_{i=1}^{r}E_i)
        \end{split}
    \end{equation}
    Vamos provar que a equação abaixo é verdadeira
    \begin{equation}
    \notag
        \begin{split}
            Pr\left(\bigcup_{i = 1}^{r+1}E_i\right) =& \sum\limits_{i=1}^{r+1}Pr(E_i)\\
            & -\sum\limits_{1\leq i<j\leq r+1}^{r}Pr(E_i \cap E_j)\\
            & +\sum\limits_{1\leq i<j<k\leq r+1}^{r}Pr(E_i \cap E_j \cap E_k)\\
            & \ldots\\
            & +(-1)^{(r+1)-1}Pr\left(\bigcap_{i=1}^{r+1}E_i\right)
        \end{split}
    \end{equation}

    \textbf{Passo} : quero provar que para $k = n + 1$ a o hipótese segura para $r + 1$

    \hspace*{30pt} Assim temos \newline
    \begin{equation}
    \label{eq:dev}
        \begin{split}
            Pr\left(\bigcup_{i = 1}^{r+1}E_i\right) =& Pr\left(\bigcup_{i = 1}^{r+1}E_i \cup E_{r+1}\right)\\
            =& Pr\left(\bigcup_{i = 1}^{r+1}E_i) + Pr(E_{r+1}\right) - Pr\left(\bigcup_{i = 1}^{r+1}E_i \cup E_{r+1}\right)
        \end{split}
    \end{equation}
    Considerando 
    \begin{equation}
        \notag
        Pr\left(\bigcup_{i = 1}^{r+1}E_i \cup E_{r+1}\right)
    \end{equation}
    Podemos reescrever usando o fato \textit{Intersecção Distribuida sobre a União}, ficamos com

    \begin{equation}
        \notag
        Pr\left(\bigcup_{i = 1}^{r+1}(E_i \cup E_{r+1})\right)
    \end{equation}
    Sendo assim, podemos aplicar a hipótese de indução
    \begin{equation}
    \notag
        \begin{split}
            Pr\left(\bigcup_{i = 1}^{r+1}(E_i \cup E_{r+1})\right) =& \sum\limits_{i=1}^{r}Pr(E_i \cap E_{r+1})\\
            & -\sum\limits_{1\leq i<j\leq r}^{r}Pr(E_i \cap E_j \cap E_{r+1})\\
            & +\sum\limits_{1\leq i<j<k\leq r}^{r}Pr(E_i \cap E_j \cap E_k \cap E_{r+1})\\
            & \ldots\\
            & +(-1)^{r-1}Pr\left(\bigcap_{i=1}^{r}E_i \cap E_{r+1}\right)
        \end{split}
    \end{equation}
    Substituindo na equação \ref{eq:dev}, temos
    \begin{equation}
    \notag
        \begin{split}
            Pr\left(\bigcup_{i = 1}^{r+1}E_i\right)& + Pr(E_{r+1}) - \left(\sum\limits_{i=1}^{r}Pr(E_i \cap E_{r+1})\right)\\
            & +\sum\limits_{1\leq i<j\leq r}^{r}Pr(E_i \cap E_j \cap E_{r+1})\\
            & -\sum\limits_{1\leq i<j<k\leq r}^{r}Pr(E_i \cap E_j \cap E_k \cap E_{r+1})\\
            & \ldots\\
            & +(-1)^{r-1}Pr\left(\bigcap_{i=1}^{r}E_i \cap E_{r+1}\right)
        \end{split}
    \end{equation}
    Disso, podemos reescrever, substituindo $Pr\left(\bigcup_{i = 1}^{r+1}E_i) + Pr(E_{r+1}\right)$
    \begin{equation}
    \notag
        \begin{split}
            Pr\left(\bigcup_{i = 1}^{r+1}E_i\right) =& \sum\limits_{i=1}^{r}Pr(E_i) + Pr(E_{r+1})\\
            & -\sum\limits_{1\leq i<j\leq r}^{r}Pr(E_i \cap E_j) - \left(\sum\limits_{i=1}^{r}Pr(E_i \cap E_{r+1})\right)\\
            & +\sum\limits_{1\leq i<j<k\leq r}^{r}Pr(E_i \cap E_j \cap E_k) + \left(\sum\limits_{1\leq i<j\leq r}^{r}Pr(E_i \cap E_j \cap E_{r+1})\right)\\
            & -\sum\limits_{1\leq i<j<k<l\leq r}^{r}Pr(E_i \cap E_j \cap E_k \cap E_l) - \left(\sum\limits_{1\leq i<j<k\leq r}^{r}Pr(E_i \cap E_j \cap E_k \cap E_{r+1})\right)\\
            & \ldots\\
            & +(-1)^{r}Pr\left(\bigcap_{i=1}^{r}E_i \cap E_{r+1}\right)
        \end{split}
    \end{equation}
\end{proof}

\end{document}
