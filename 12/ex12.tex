\documentclass{article}

\renewcommand{\thesection}{}
\renewcommand{\thesubsection}{\arabic{section}.\arabic{subsection}}
\makeatletter
\def\@seccntformat#1{\csname #1ignore\expandafter\endcsname\csname the#1\endcsname\quad}
\let\sectionignore\@gobbletwo
\let\latex@numberline\numberline
\def\numberline#1{\if\relax#1\relax\else\latex@numberline{#1}\fi}
\makeatother



\usepackage[utf8]{inputenc}

\title{Lista de algoritmos aleatorizados}
\author{Gustavo Higuchi}
\date{\today}

\usepackage{natbib}
\usepackage{graphicx}
\usepackage{amssymb}
\usepackage{amsthm}
\usepackage{amsmath}
\usepackage{color}   %May be necessary if you want to color links
\usepackage[portuguese, ruled, linesnumbered]{algorithm2e}
\usepackage[normalem]{ulem}

\usepackage{mathtools}
\DeclarePairedDelimiter\ceil{\lceil}{\rceil}
\DeclarePairedDelimiter\floor{\lfloor}{\rfloor}

% usado para linkar cada section na tabela de conteúdo com a respectiva
% página no documento
\usepackage{hyperref}
\hypersetup{
    colorlinks,
    citecolor=black,
    filecolor=black,
    linkcolor=black,
    urlcolor=black,
    linktoc=all
}

%o começo do documento
\begin{document}

% compila o título
\maketitle

% compila a tabela de conteúdos
\tableofcontents
\newpage


\chapter{}
\section{Exercício 1}
Seja $X_i$ a v.a. se Bob ganhau o i-ésimo jogo, para um $1\leq i \leq n$, então\\

\begin{equation}
	\notag
	Pr[X_i = 1] = 0,4
\end{equation}

e para $X = \sum\limits_{i=1}^n X_i$,
\begin{equation}
	\notag
	E[X] = \sum\limits_{i=1}^n E[X_i] = \sum\limits_{i=1}^n Pr[X_i = 1] = \sum\limits_{i=1}^n 0,4 = 0,4n
\end{equation}

Para Alice perder, Bob deve ganhar mais da metade dos jogos. Pela \emph{desigualdade de Chernoff}
\begin{equation}
	\notag
	Pr[X > (1 + \delta)\mu] \leq e^{-\mu \delta^2 / 3}\text{, para um }0 < \delta \leq 1
\end{equation}

Então, temos

\begin{equation}
	\notag
	\begin{split}
		Pr[X > 0,5n] = Pr[X > (1 + 0,25)0,4n] &\leq e^{-\mu \delta / 3}\\
		&= e^{-0,4n (0,25)^2 / 3} \\
		&= e^{-0,4n 0,0125 / 3}\\
		&= e^{-0,025n / 3}
	\end{split}
\end{equation}

\section{Exercício 2}
Seja $X$ o número de vezes em que um dado deu 6 em $n$ jogadas.

Seja $p$ a probabilidade de $X \geq \dfrac{n}{4}$.

Seja $X_i$ a v.a. correspondente a i-ésima jogada do dado e $X_i = 1$ se o dado der 6 e $X_i = 0$ caso contrário.

\begin{equation}
	\notag
	E[X] = \sum\limits_{i=1}^nE[X_i] = \sum\limits_{i=1}^n Pr[X_i = 1] =  \sum\limits_{i=1}^n \dfrac{1}{6} = \dfrac{n}{6}
\end{equation}

Por \emph{Markov}, temos

\begin{equation}
	\begin{split}
		Pr\left(X \geq \dfrac{n}{4}\right) &\leq \dfrac{E[X]}{n/4}\\
		&= \dfrac{n/6}{n/4}\\
		&= \dfrac{4n}{6n} = \dfrac{2}{3}
	\end{split}
\end{equation}

Por \emph{Chebyshev}, temos
\begin{equation}
\notag
	E[X^2] = \sum\limits_{i=1}^nE[X_i^2] = \sum\limits_{i=1}^n\dfrac{1}{6} = \dfrac{n}{6}
\end{equation}

então

\begin{equation}
	\begin{split}
		Pr\left[X \geq \dfrac{n}{4}\right] &= Pr\left[|X - \dfrac{n}{6}|\geq \dfrac{n}{4} - \dfrac{n}{6}\right]\\
		&= Pr\left[|X - \dfrac{n}{6}|\geq \dfrac{n}{12}\right]\\
		&\leq \dfrac{Var(X)}{(n/12)^2}\\
		&=\dfrac{E[X^2] - (E[X])^2}{n/144}\\
		&= \dfrac{n/6 - n/36}{n/144}\\
		&= \dfrac{20}{n}
	\end{split}
\end{equation}

Por \emph{Chernoff}, temos
\begin{equation}
	\begin{split}
		Pr\left[X \geq \dfrac{n}{4} \right] &= Pr\left[X \geq (1+0,5)\dfrac{n}{6} \right]\\
		&\leq e^{-n/6(0,5)^2/3}\\
		&= e^{-n/72}
	\end{split}
\end{equation}

\section{Exercício 3}
Seja $X$ o número total de caras e $X_i$ a v.a. que representa a chance da i-ésima 
jogada der cara, temos\\

\begin{equation}
	\notag
	Pr[X_i = 1] = \dfrac{1}{2}
\end{equation}
e
\begin{equation}
	\notag
	E[X] = \sum\limits_{i=1}^{100}E[X_i] = \sum\limits_{i=1}^{100}\dfrac{1}{2} = 50
\end{equation}

Levando em conta a \emph{desigualdade de Chernoff}

\begin{equation}
	\notag
	Pr[X\geq (1+\delta)\mu] \leq e^{-\mu \delta / 3}\text{, para um }0 \leq \delta \leq 1
\end{equation}
e $\mu = E[X]$.

Temos
\begin{equation}
	\begin{split}
		Pr[X \geq (1+0,1)50] &\leq e^{-50(0,1)^2/3}\\
		&= e^{-0,5/3}\\
		&= e^{-1/6}
	\end{split}
\end{equation}

E para um $n = 1000$, vamos ter

\begin{equation}
	\begin{split}
		Pr[X \geq (1+0,1)500] &\leq e^{-500(0,1)^2/3}\\
		&= e^{-5/3}
	\end{split}
\end{equation}

\end{document}

