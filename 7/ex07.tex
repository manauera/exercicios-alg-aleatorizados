\documentclass{article}

\renewcommand{\thesection}{}
\renewcommand{\thesubsection}{\arabic{section}.\arabic{subsection}}
\makeatletter
\def\@seccntformat#1{\csname #1ignore\expandafter\endcsname\csname the#1\endcsname\quad}
\let\sectionignore\@gobbletwo
\let\latex@numberline\numberline
\def\numberline#1{\if\relax#1\relax\else\latex@numberline{#1}\fi}
\makeatother



\usepackage[utf8]{inputenc}

\title{Lista de Exercícios 7}
\author{Gustavo Higuchi}
\date{\today}

\usepackage{natbib}
\usepackage{graphicx}
\usepackage{amssymb}
\usepackage{amsthm}
\usepackage{amsmath}
\usepackage{color}   %May be necessary if you want to color links
\usepackage[portuguese, ruled, linesnumbered]{algorithm2e}
\usepackage[normalem]{ulem}

\usepackage{mathtools}
\DeclarePairedDelimiter\ceil{\lceil}{\rceil}
\DeclarePairedDelimiter\floor{\lfloor}{\rfloor}

% usado para linkar cada section na tabela de conteúdo com a respectiva
% página no documento
\usepackage{hyperref}
\hypersetup{
    colorlinks,
    citecolor=black,
    filecolor=black,
    linkcolor=black,
    urlcolor=black,
    linktoc=all
}

%o começo do documento
\begin{document}

% compila o título
\maketitle

% compila a tabela de conteúdos
\tableofcontents
\newpage


\chapter{}
\section{Exercício 1}
\hspace*{15pt}É dito sem memória porque o sucesso do futuro independe dos fracassos anteriores.\\

Em outras palavras, a probabilidade de se obter um primeiro sucesso na n-ésima tentativa
não depende do número de fracassos observados.\\

Formalmente dado pelo \textit{lemma} 2.8, onde \textit{para uma variável aleatória X com parametro p
e para um $n > 0$} 
\begin{equation}
    \notag
    Pr(X = n + k | X > k) = Pr(X = n)
\end{equation}


\section{Exercício 2}
\begin{proof}
    Para um $0 < p < 1$
    \begin{equation}
    \notag
        \begin{split}
        \sum\limits_{i\geq1}Pr(X = i) &= \sum\limits_{i\geq1}(1-p)^{i-1}p\\
        &= p\sum\limits_{i\geq1}(1-p)^{i-1}\\
        &= p(1 + (1-p) + (1-p)^2 + (1-p)^3 + \ldots)\\
        \text{dado a série geométrica, temos}&\\
        &= p\left(\dfrac{1}{1-(1-p)}\right) = p\left(\dfrac{1}{p}\right) = 1
        \end{split}
    \end{equation}
\end{proof}

\section{Exercício 3}
\hspace*{15pt}Para $k=1$ temos $Pr(\text{ser guardado}) = 1/1 $, então na primeira
temos apenas um número, que tem $Pr(\text{ser guardado}) = 1$. \\

Na segunda iteração, para $k=2$ com $Pr(\text{ser guardado}) = 1/2$, para
temos 2 números.\\

Na terceira, para $k=3$ com $Pr(\text{ser guardado}) = 1/3$, para
temos 3 números. Então a escolha anterior com $Pr(\text{ser guardado}) = 1/2$,
ficamos com um de dois números escolhidos de forma uniforme que será substituido
com $Pr(\text{ser substituido}) = 1/3$.


\end{document}
