\documentclass{article}

\renewcommand{\thesection}{}
\renewcommand{\thesubsection}{\arabic{section}.\arabic{subsection}}
\makeatletter
\def\@seccntformat#1{\csname #1ignore\expandafter\endcsname\csname the#1\endcsname\quad}
\let\sectionignore\@gobbletwo
\let\latex@numberline\numberline
\def\numberline#1{\if\relax#1\relax\else\latex@numberline{#1}\fi}
\makeatother



\usepackage[utf8]{inputenc}

\title{Lista de Exercícios de Algoritmos Aleatorizados}
\author{Gustavo Higuchi}
\date{\today}

\usepackage{natbib}
\usepackage{graphicx}
\usepackage{amssymb}
\usepackage{amsthm}
\usepackage{amsmath}
\usepackage{color}   %May be necessary if you want to color links
\usepackage[portuguese, ruled, linesnumbered]{algorithm2e}
\usepackage[normalem]{ulem}

\usepackage{mathtools}
\DeclarePairedDelimiter\ceil{\lceil}{\rceil}
\DeclarePairedDelimiter\floor{\lfloor}{\rfloor}

% usado para linkar cada section na tabela de conteúdo com a respectiva
% página no documento
\usepackage{hyperref}
\hypersetup{
    colorlinks,
    citecolor=black,
    filecolor=black,
    linkcolor=black,
    urlcolor=black,
    linktoc=all
}

%o começo do documento
\begin{document}

% compila o título
\maketitle

% compila a tabela de conteúdos
\tableofcontents
\newpage


\chapter{}
\section{Exercício 1}
\begin{proof}
	\begin{equation}
	\notag
		\begin{split}
			E[X_1 + X_2 | Y = y] &= \sum\limits_{i}\sum\limits_{j}(i+j) Pr((X_1 = i \cap X_2 = j)| Y = y)\\
			&= \sum\limits_{i}\sum\limits_{j}i Pr((X_1 = i \cap X_2 = j)| Y = y) + \sum\limits_{i}\sum\limits_{j}j Pr((X_1 = i \cap X_2 = j)| Y = y)\\
			&= \sum\limits_{i} i \sum\limits_{j}Pr((X_1 = i \cap X_2 = j)| Y = y) + \sum\limits_{j} j \sum\limits_{i}Pr((X_1 = i \cap X_2 = j)| Y = y)\\
			&= \sum\limits_{i}i Pr(X_1 = i | Y = y) + \sum\limits_{j}j Pr(X_2 = j| Y = y)\\
			&= E[X_1 | Y = y] + E[X_2 | Y = y]
		\end{split}
	\end{equation}
\end{proof}

\section{Exercício 2}
Dado que a esperança de processos gerados total é igual a $(np)^i$, se tivermos
um $np \geq 1$, teremos uma função sempre crescente sem limites, uma vez que para
cada processo, seria gerado mais $n$ processos, e assim por diante.\\

Enquanto que se tivermos um $np < 1$ teremos um somatório de um número cada vez
menor, se aproximando de zero. Então por isso que nesse caso a função é limitada 
em $\dfrac{1}{1-np}$ 

\section{Exercício 3}
\begin{proof}
	Dado que temos 
	\begin{equation}
	\notag
		\begin{split}
			Pr(\text{2 cópias geradas}) &= p\\
			Pr(\text{nenhum cópia gerada}) &= 1-p
		\end{split}
	\end{equation}
	Queremos saber qual é o número TOTAL de cópias geradas na i-ésima iteração.\\

	Para um processo qualquer, temos que a esperança de processos gerados é:
	\begin{equation}
	\notag
		\begin{split}
			Y_1 &= \sum\limits_{j = 1}^{2}Pr(\text{2 cópias geradas})\\
			&= 2p
		\end{split}
	\end{equation}

	Então, se supormos que sabemos o número de processos gerados na iteração anterior, 
	i.e. $Y_{i-1} = y_{i-1}$, significa que teremos $y_{i-1}$ processos. Então,
	seja $Z_k$ o número de processos gerados pelo k-ésimo processo, sendo assim
	temos o seguinte:\\

	\begin{equation}
	\notag
		\begin{split}
			E[Y_i | Y_{i-1}] & = E\left[\sum\limits_{k=1}^{y_{i-1}}Z_k | Y_{i-1} = y_{i-1}\right]\\
			\text{pela Lin. da Esperança}&\\
			&= \sum\limits_{k=1}^{y_{i-1}}E[Z_k | Y_{i-1} = y_{i-1}]\\
			\text{pela definição de esperança condicional}&\\
			&= \sum\limits_{k=1}^{y_{i-1}}\sum\limits_j j . Pr(Z_k = j | Y_{i-1} = y_{i-1})
		\end{split}
	\end{equation}

	A probabilidade do k-ésimo processo gerar ou não novos processos não depende de quantos
	processos foram gerados na iteração anterior, então temos

	\begin{equation}
	\notag
		\begin{split}
			&=\sum\limits_{k=1}^{y_{i-1}}\sum\limits_j j . Pr(Z_k = j)\\
			\text{pela definição de esperaça}&\\
			&= \sum\limits_{k=1}^{y_{i-1}}E[Z_k]\\
			&= \sum\limits_{k=1}^{y_{i-1}}2p = y_{i-1}2p
		\end{split}
	\end{equation}

	Assim, pelo teorema 2.7 do MU
	\begin{equation}
	\notag
		\begin{split}
			E[Y_i] &= E[E[Y_i|Y_{i-1}]]\\
			&= \sum\limits_{y_{i-1} \geq 0} E[Y_i | Y_{i-1} = y_{i-1}] . Pr(Y_{i-1} = y_{i-1})\\
			&= \sum\limits_{y_{i-1} \geq 0} y_{i-1}.2p . Pr(Y_{i-1} = y_{i-1})\\
			&= 2p \sum\limits_{y_{i-1} \geq 0} y_{i-1} . Pr(Y_{i-1} = y_{i-1})\\
			\text{pela definição de esperança}&\\
			&= 2p E[Y_{i-1}] = (2p)^i
		\end{split}
	\end{equation}

	E portanto
	\begin{equation}
	\notag
		\begin{split}
			E\left[\sum\limits_{i\geq0}Y_i\right] &= \sum\limits_{i\geq0}E[Y_i]\\
			&= \sum\limits_{i\geq0}(2p)^i
		\end{split}
	\end{equation}

	\begin{equation}
	\notag
		E\left[\sum\limits_{i\geq0}Y_i\right] =\begin{cases}
        \infty, & \text{se $2*p \geq 1$}\\
        \dfrac{1}{i-2p}, &\text{se $2*p < 1$}
    \end{cases}
	\end{equation}
\end{proof}

\section{Exercício 4}
\begin{proof}
	A idéia aqui é que cada pensar em quanto que cada vértice contribui para o
	número de ciclos esperado. Se $v$ está em um ciclo de tamanho $l$, ele 
	contribui com $\dfrac{1}{l}$ para a esperança, então pela Lin. da Esperança
	\begin{equation}
	\notag
		\begin{split}
			E[\text{número de ciclos}] &= \sum\limits_{v=1}^{n}\sum\limits_{l=1}^{n} Pr(v\text{ está em ciclo de tamanho }l).\dfrac{1}{l}\\
			&= \sum\limits_{v=1}^{n}\sum\limits_{l=1}^{n} \dfrac{1}{nl}\\
			\text{invertendo a ordem dos somatórios}&\\
			&= \sum\limits_{l=1}^{n}\sum\limits_{v=1}^{n} \dfrac{1}{nl}\\
			&= \sum\limits_{l=1}^{n} n * \dfrac{1}{nl} \\
			E[\text{número de ciclos}]&= \sum\limits_{l=1}^{n} \dfrac{1}{l} = H_n
		\end{split}
	\end{equation}
\end{proof}
\end{document}
