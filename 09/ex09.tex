\documentclass{article}

\renewcommand{\thesection}{}
\renewcommand{\thesubsection}{\arabic{section}.\arabic{subsection}}
\makeatletter
\def\@seccntformat#1{\csname #1ignore\expandafter\endcsname\csname the#1\endcsname\quad}
\let\sectionignore\@gobbletwo
\let\latex@numberline\numberline
\def\numberline#1{\if\relax#1\relax\else\latex@numberline{#1}\fi}
\makeatother



\usepackage[utf8]{inputenc}

\title{Lista de Exercícios 9}
\author{Gustavo Higuchi}
\date{\today}

\usepackage{natbib}
\usepackage{graphicx}
\usepackage{amssymb}
\usepackage{amsthm}
\usepackage{amsmath}
\usepackage{color}   %May be necessary if you want to color links
\usepackage[portuguese, ruled, linesnumbered]{algorithm2e}
\usepackage[normalem]{ulem}

\usepackage{mathtools}
\DeclarePairedDelimiter\ceil{\lceil}{\rceil}
\DeclarePairedDelimiter\floor{\lfloor}{\rfloor}

% usado para linkar cada section na tabela de conteúdo com a respectiva
% página no documento
\usepackage{hyperref}
\hypersetup{
    colorlinks,
    citecolor=black,
    filecolor=black,
    linkcolor=black,
    urlcolor=black,
    linktoc=all
}

%o começo do documento
\begin{document}

% compila o título
\maketitle

% compila a tabela de conteúdos
\tableofcontents
\newpage


\chapter{}
\section{Exercício 1}
\hspace*{15pt}
\begin{equation}
	\begin{split}
		Var[X] &= E[(X-E[X])^2]\\
		\text{pela Def 3.2 do MU}&\\
		&= E[X^2] - (E[X])^2\\
		&= \sum\limits_{i=1}^{n}\dfrac{1}{n}.i^2 - \left(\sum\limits_{i=1}^{n}\dfrac{1}{n}.i\right)^2\\
		&= \dfrac{(n+1)(2n+1)}{6} - \left(\dfrac{n+1}{2}\right)^2\\
		&= \dfrac{2n^2 + 4n + 1}{6} - \dfrac{n^2 + 2n + 1}{4}\\
		&= \dfrac{2(2n^2 + 4n + 1) - 3(n^2 + 2n + 1)}{12}\\
		&= \dfrac{4n^2 + 8n + 2 - 3n^2 - 6n - 3}{12}\\
		&= \dfrac{n^2 + 2n - 1}{12}\\
		&= \dfrac{(n+1)(n-1)}{12}
	\end{split}
\end{equation}

\section{Exercício 2}
\begin{equation}
	\begin{split}
		Var[X] &= E[X^2] - (E[X])^2\\
		&= 2\sum\limits_{i=-k}^{k}\dfrac{1}{2k+1}.i^2 - \sum\limits_{i=-k}^{k}\dfrac{1}{2k+1}\\
		&= \dfrac{k(k+1)}{3} - 0 \\
		&= \dfrac{k(k+1)}{3}
	\end{split}
\end{equation}

\section{Exercício 3}
\begin{proof}
\begin{equation}
	\begin{split}
		Var[cX] &= E[(cX)^2] - (E[cX])^2\\
		&= E[c^2X^2] - (cE[X])^2(\text{pela lin. da esperança})\\
		&= c^2E[X^2] - c^2 (E[X]^2) (\text{de novo pela lin. da esperança})\\
		&= c^2(E[X^2] - (E[X])^2)\\
		&= c^2Var[X]
	\end{split}
\end{equation}
\end{proof}

\section{Exercício 4}
\begin{proof}
	\hspace*{15pt}Pela definição,
	\begin{equation}
		\notag
		Pr(X \geq a) = \dfrac{E[X]}{a}
	\end{equation}
	para um $a > 0$.\\

	Se temos $a = kE[X]$, então
	\begin{equation}
		\notag
		Pr(X \geq kE[X]) = \dfrac{E[X]}{kE[X]} = \dfrac{1}{k}
	\end{equation}
	para uma v.a. X que assuma apenas valores não negativos.
\end{proof}
\end{document}
