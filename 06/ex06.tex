\documentclass{article}

\renewcommand{\thesection}{}
\renewcommand{\thesubsection}{\arabic{section}.\arabic{subsection}}
\makeatletter
\def\@seccntformat#1{\csname #1ignore\expandafter\endcsname\csname the#1\endcsname\quad}
\let\sectionignore\@gobbletwo
\let\latex@numberline\numberline
\def\numberline#1{\if\relax#1\relax\else\latex@numberline{#1}\fi}
\makeatother



\usepackage[utf8]{inputenc}

\title{Lista de Exercícios 5}
\author{Gustavo Higuchi}
\date{\today}

\usepackage{natbib}
\usepackage{graphicx}
\usepackage{amssymb}
\usepackage{amsthm}
\usepackage{amsmath}
\usepackage{color}   %May be necessary if you want to color links
\usepackage[portuguese, ruled, linesnumbered]{algorithm2e}
\usepackage[normalem]{ulem}

\usepackage{mathtools}
\DeclarePairedDelimiter\ceil{\lceil}{\rceil}
\DeclarePairedDelimiter\floor{\lfloor}{\rfloor}

% usado para linkar cada section na tabela de conteúdo com a respectiva
% página no documento
\usepackage{hyperref}
\hypersetup{
    colorlinks,
    citecolor=black,
    filecolor=black,
    linkcolor=black,
    urlcolor=black,
    linktoc=all
}

%o começo do documento
\begin{document}

% compila o título
\maketitle

% compila a tabela de conteúdos
\tableofcontents
\newpage


\chapter{}
\section{Exercício 1}
\begin{proof}
    Do teorema binomial, temos
    \begin{equation}
        \notag
        (x + y)^n = \sum\limits_{j=0}^{n} {n \choose j}x^j y^{n-j}
    \end{equation}

    Então, se definirmos $x = p$ e $y = 1 - p$, temos o seguinte

    \begin{equation}
        \notag
        \begin{split}
            (p + 1-p)^n &= \sum\limits_{j=0}^{n} {n \choose j}p^j (1-p)^{n-j}\\
            1^n &= \sum\limits_{j=0}^{n} {n \choose j}p^j (1-p)^{n-j}
        \end{split}
    \end{equation}

    Como $1^n$ é sempre $1$, podemos afirmar que a soma de todas as probabilidades
    de uma variável aleatória binomial é também $1$. 
\end{proof}

\section{Exercício 2}
O exemplo mais trivial que temos é quando queremos saber a probabilidade de 
acertar um número $k$ de caras, numa quantidade de $n$ lances de moeda.

\section{Exercício 3}
\begin{proof}
    \begin{equation}
        \notag
        \begin{split}
            E[X] &= \sum\limits_{j} j {n \choose j} p^j(1-p)^{n-j}\\
            &= \sum\limits_{j} \dfrac{n!}{(j-1)!(n-j)!} p^j(1-p)^{n-j}\\
            &= n.p \sum\limits_{j} \dfrac{(n-1)!}{(j-1)!((n-1)-(j-1)!} p^{j-1}(1-p)^{(n-1)-(j-1)}\\
            &= n.p \sum\limits_{k}^{n-1} {n-1 \choose k} p^k(1-p)^{(n-1)-k}\\
            &= n.p
        \end{split}
    \end{equation}
\end{proof}

\section{Exercício 4}
\subsection*{(a)}
Temos
\begin{equation}
    \notag
    Pr(X = 1) = {n \choose 1}p(1-p)^{n-1} = \dfrac{n!}{(n-1)!}p(1-p)^{n-1} = np(1-p)^{n-1}
\end{equation}

\subsection*{(b)}
Considerando que não temos ninguém positivo ou pelo menos uma pessoa, ficamos com o seguinte
\begin{equation}
    \notag
    E[X] = \dfrac{n}{k} * np(1-p)^{n-1}
\end{equation}

\subsection*{(c)}
Escolher um número $k$ de pessoas tal que $k * p < 1$. 

Ou seja, que tenha menos de uma pessoa por grupo de usuários, caso contrário,
teremos pelo menos uma pessoa por grupo, e isso significaria que terá de testar 
individualmente para todos os usuários, além dos testes do grupo.

\subsection*{(d)}
Levando em consideração que temos uma probabilidade ideal de termos uma pessoa
positiva no grupo, ou seja, que seja como na resposta anterior (c). Podemos afirmar
que teremos pelo menos um grupo com todas as k pessoas positivas. Então, se possuirmos 
$\dfrac{n}{k}$ grupos, ficamos com
\begin{equation}
    \notag
    (k-1) * (\dfrac{n - k}{k}) \leq k * n
\end{equation}
\end{document}
