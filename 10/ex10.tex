\documentclass{article}

\renewcommand{\thesection}{}
\renewcommand{\thesubsection}{\arabic{section}.\arabic{subsection}}
\makeatletter
\def\@seccntformat#1{\csname #1ignore\expandafter\endcsname\csname the#1\endcsname\quad}
\let\sectionignore\@gobbletwo
\let\latex@numberline\numberline
\def\numberline#1{\if\relax#1\relax\else\latex@numberline{#1}\fi}
\makeatother



\usepackage[utf8]{inputenc}

\title{Lista de Exercícios 12}
\author{Gustavo Higuchi}
\date{\today}

\usepackage{natbib}
\usepackage{graphicx}
\usepackage{amssymb}
\usepackage{amsthm}
\usepackage{amsmath}
\usepackage{color}   %May be necessary if you want to color links
\usepackage[portuguese, ruled, linesnumbered]{algorithm2e}
\usepackage[normalem]{ulem}

\usepackage{mathtools}
\DeclarePairedDelimiter\ceil{\lceil}{\rceil}
\DeclarePairedDelimiter\floor{\lfloor}{\rfloor}

% usado para linkar cada section na tabela de conteúdo com a respectiva
% página no documento
\usepackage{hyperref}
\hypersetup{
    colorlinks,
    citecolor=black,
    filecolor=black,
    linkcolor=black,
    urlcolor=black,
    linktoc=all
}

%o começo do documento
\begin{document}

% compila o título
\maketitle

% compila a tabela de conteúdos
\tableofcontents
\newpage


\chapter{}
\section{Exercício 1}
Seja $X_i$ o número dado pelo dado $i$ e $X$ a soma de todas as v.a. $X_i$, ou seja

\begin{equation}
	\notag
	X = \sum\limits_{i=1}^{100} X_i
\end{equation}

Então, 
\begin{equation}
	\notag
	E[X_i] = \sum\limits_{j=1}^6 jPr(X_i = j) = \sum\limits_{j=1}^6 j\dfrac{1}{6} = \dfrac{7}{2}
\end{equation}
e disso,
\begin{equation}
	\notag
	E[X] = \sum\limits_{j=1}^{100} E[X_i] = \sum\limits_{j=1}^{100} 3,5 = 350
\end{equation}

mais adiante,
\begin{equation}
	\notag
	\begin{split}
		E[X_i^2] &= \sum\limits_{j=1}^{6}j^2Pr(X_i = j)\\
		&= \sum\limits_{j=1}^{6}j^2 \dfrac{1}{6}\\
		\text{usando } \sum_{i=1}^ni^2 = n(n+1)(2n+1)/6 &\\
		&= \dfrac{1}{6} . \dfrac{6(7)(13)}{6}\\
		&= \dfrac{91}{6}
	\end{split}
\end{equation}

Último passo é calcular a variancia de $X$, então

\begin{equation}
\notag
	\begin{split}
		Var(X) &= Var(\sum\limits_{i=1}^{100}X_i)\\
		\text{pelo teo 3.3 do MU} &\\
		&= \sum\limits_{i=1}^{100}Var(X_i)\\
		&= \sum\limits_{i=1}^{100}E[X_i^2] - E[X]^2\\
		&= \sum\limits_{i=1}^{100}\dfrac{91}{6} - (\dfrac{7}{2})^2\\
		&= \sum\limits_{i=1}^{100}\dfrac{35}{12}
	\end{split}
\end{equation}

Finalmente, temos

\begin{equation}
	\notag
	Pr(|X - 350|\geq50) \leq \dfrac{\sum\limits_{i=1}^{100} 35/12}{50^2} = \dfrac{100 . 35/12}{50^2} = \dfrac{7}{60}
\end{equation}

\section{Exercício 2}
\subsection*{(a)}
Seja $Y = (X - E[X])^k$.\\

Pela desigualdade de Markov, temos
\begin{equation}
	\label{eq:2a-1}
	Pr(Y \geq t^kE[Y]) \geq \dfrac{E[Y]}{t^k E[Y]} = \dfrac{1}{t^k}
\end{equation}

então,
\begin{equation}
	\notag
	\begin{split}
		Pr(Y \geq t^kE[Y]) &= Pr(\sqrt[k]{Y} \geq t \sqrt[k]{E[Y]})\\
		&= Pr(|X - E[X]| \geq t \sqrt[k]{E[(X - E[X])^k]})
	\end{split}
\end{equation}

Como já tínhamos calculado em \ref{eq:2a-1},
\begin{equation}
	\notag
	Pr(|X - E[X| \geq t \sqrt[k]{E[(X - E[X])^k]} ) \leq \dfrac{1}{t^k}
\end{equation}

\subsection*{(b)}
Como $X$ pode ser qualquer variável aleatória, o valor $(X - E[X])^k$ para um $k$
ímpar pode ser um número negativo, e portanto a \emph{desigualdade de Markov} não se aplica.

\section{Exercício 3}
\subsection*{(a)}
Dado $b_i$ e $b_j$ o par de bits da v.a. $Y_i$, então
\begin{equation}
	\notag
	Y_i =\begin{cases}
		1, \text{ se }b_i = b_j\\
		0, \text{ caso contrário}
	\end{cases}
\end{equation}

Dado que nosso espaço amostral é
\begin{equation}
	\notag
	\Omega = \{(0,0), (0,1), (1,0), (1,1) \}
\end{equation}

então temos
\begin{equation}
	\notag
	\begin{split}
		Pr(Y_i = 1) &= \dfrac{2}{8} = \dfrac{1}{4}\\
		\text{e}&\\
		Pr(Y_i = 0) &= \dfrac{2}{8} = \dfrac{1}{4}
	\end{split}
\end{equation}

\subsection*{(b)}
Seja $b_1, b_2, \ldots , b_n$ os $n$ bits aleatórios. Seja também $\oplus$ símbolo
para o \emph{ou-exclusivo} e $Y_1 = b_1 \oplus b_2$, $Y_2 = b_2 \oplus b_3$ e $Y_3 = b_3 \oplus b_1$. Temos o seguinte

\begin{equation}
	\notag
	Pr(Y_1 = 1 \cup Y_2 = 1 \cup Y_3 = 1) = 0 \neq Pr(Y_1 = 1). Pr(Y_2 = 1). Pr(Y_3 = 1)
\end{equation}
uma vez que se $Y_1 = 1$ e $Y_2 = 1$, então $b_1 = b_3$ e assim $Y_3 = 0$. Portanto, as
v.a. $Y_i$ não são mutalmente independentes.

\subsection*{(c)}
Caso $Y_i$ e $Y_j$ não compartilhem nenhum bit, eles são independentes e pelo teorema 3.3 de MU, $E[Y_iY_j] = E[Y_i]E[Y_j]$.

Caso $Y_i$ e $Y_j$ compartilhem um bit, temos
\begin{equation}
	\notag
	\begin{split}
		Y_i &= b_1 \oplus b_2\\
		\text{ e }&\\
		Y_j &= b_2 \oplus b_3
	\end{split}
\end{equation}

Enumerando todos os possíveis valores para os bits $b_1$, $b_2$ e $b_3$, temos
\begin{equation}
	\notag
	Pr(Y_i = 1 \cup Y_j = 1) = \dfrac{2}{8} = \dfrac{1}{4}
\end{equation}

Então, 
\begin{equation}
	\notag
	\begin{split}
		E[Y_iY_j] &= 1. Pr(Y_i = 1 \cup Y_j = 1)\\
		&= 1.\dfrac{1}{4}\\
		&= E[Y_i]E[Y_j]
	\end{split}
\end{equation}

\subsection*{(d)}
Em 3.15 do MU diz que dado uma v.a. $X = \sum\limits_{i=1}^n X_i$, então
$E[X_iX_j] = E[X_i]E[X_j]$ para cada $i$ e $j$ t.q. $1 \leq i < j \leq n$,
$Var(X) = \sum\limits_{i=1}^nVar(X_i)$.

Então, 
\begin{equation}
	\notag
	\begin{split}
		Var[Y] &= Var\left[\sum\limits_{i=1}^{m}Y_i \right]\\
		&= \sum\limits_{i=1}^{m}Var\left[Y_i \right]\\
		&= \sum\limits_{i=1}^{m}\left(E[Y_i^2] - E[Y_i]^2 \right)\\
		&= m. \left(\dfrac{1}{2} - \dfrac{1}{4} \right)\\
		&= m . \dfrac{1}{4}\\
		&= \dfrac{{n \choose 2}}{4}
	\end{split}
\end{equation}

\subsection*{(e)}
\begin{proof}
	\begin{equation}
		\notag
		\begin{split}
			Pr(|Y - E[Y]| \geq n) &= \dfrac{Var(Y)}{n^2}\\
			&= \dfrac{\dfrac{{n \choose 2}}{4}}{n^2}\\
			&= \dfrac{n(n-1)/8}{n^2}\\
			&= \dfrac{n(n-1)}{8n^2}\\
			&= \dfrac{1}{8} - \dfrac{1}{8n}
		\end{split}
	\end{equation}

	Sendo assim, 
	\begin{equation}
		\notag
		Pr(|Y - E[Y]| \geq n) = \dfrac{1}{8} - \Omega(n^{-1})
	\end{equation}
\end{proof}
\end{document}
