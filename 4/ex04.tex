\documentclass{article}

\renewcommand{\thesection}{}
\renewcommand{\thesubsection}{\arabic{section}.\arabic{subsection}}
\makeatletter
\def\@seccntformat#1{\csname #1ignore\expandafter\endcsname\csname the#1\endcsname\quad}
\let\sectionignore\@gobbletwo
\let\latex@numberline\numberline
\def\numberline#1{\if\relax#1\relax\else\latex@numberline{#1}\fi}
\makeatother



\usepackage[utf8]{inputenc}

\title{Lista de Exercícios 12}
\author{Gustavo Higuchi}
\date{\today}

\usepackage{natbib}
\usepackage{graphicx}
\usepackage{amssymb}
\usepackage{amsthm}
\usepackage{amsmath}
\usepackage{color}   %May be necessary if you want to color links
\usepackage[portuguese, ruled, linesnumbered]{algorithm2e}
\usepackage[normalem]{ulem}

\usepackage{mathtools}
\DeclarePairedDelimiter\ceil{\lceil}{\rceil}
\DeclarePairedDelimiter\floor{\lfloor}{\rfloor}

% usado para linkar cada section na tabela de conteúdo com a respectiva
% página no documento
\usepackage{hyperref}
\hypersetup{
    colorlinks,
    citecolor=black,
    filecolor=black,
    linkcolor=black,
    urlcolor=black,
    linktoc=all
}

%o começo do documento
\begin{document}

% compila o título
\maketitle

% compila a tabela de conteúdos
\tableofcontents
\newpage


\chapter{}
\section{Exercício 1}
\textbf{To improve the probability of success of the randomized min-cut algo-
rithm, it can be run multiple times.}
\subsection*{(a)} 
\textbf{Consider running the algorithm twice. Determine the number of edge contractions
and bound the probability of finding a min-cut.}\\

A quantidade esperada de arestas contraídas na execução do algoritmo é de pelo menos $\frac{nk}{2} - k$,
onde $n$ é a quantidade de vértices e $k$ o corte mínimo. Assumindo que as últimas $k$
arestas não serão contraídas por estarem no corte mínimo.\\

A probabilidade de achar o corte mínimo executando o algoritmo aleatorizado duas vezes 
é de pelo menos $\frac{4}{n^2(n-1)^2}$.


\subsection*{(b)} 
\textbf{Consider the following variation. Starting with a graph with n vertices, first 
contract the graph down to k vertices using the randomized min-cut algorithm. Make
copies of the graph with k vertices, and now run the randomized algorithm on this
reduced graph $l$ times, independently. Determine the number of edge contractions
and bound the probability of finding a minimum cut.}\\

Se executarmos o CMA (Corte Mínimo Aleatorizado) para $k$ vértices independentemente
$e$ vezes, temos a probabilidade\\

\begin{equation}
\notag
    \left(\frac{2}{k(k-1)}\right)^{l+c}
\end{equation}
onde $c$ é o número de cópias.\\

E o número esperado de arestas contraídas é

\begin{equation}
    \notag
    \left(\frac{km}{2} - m\right).c
\end{equation} 
onde $c$ é o número de cópias e $m$ o corte mínimo.
\subsection*{(c)} 
\textbf{Find optimal (or at least near-optimal) values of k and t for the variation
in (b) that maximize the probability of finding a minimum cut while using the same number
of edge contractions as running the original algorithm twice.}

\section{Exercício 2}
\textbf{Generalizing on the notion of a cut-set, we define an r-way cut-set in a
graph as a set of edges whose removal breaks the graph into r or more connected 
components. Explain how the randomized min-cut algorithm can be used to find minimum
r-way cut-sets, and bound the probability that it succeeds in one iteration.}


\end{document}
