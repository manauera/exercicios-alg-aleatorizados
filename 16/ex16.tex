\documentclass{article}

\renewcommand{\thesection}{}
\renewcommand{\thesubsection}{\arabic{section}.\arabic{subsection}}
\makeatletter
\def\@seccntformat#1{\csname #1ignore\expandafter\endcsname\csname the#1\endcsname\quad}
\let\sectionignore\@gobbletwo
\let\latex@numberline\numberline
\def\numberline#1{\if\relax#1\relax\else\latex@numberline{#1}\fi}
\makeatother



\usepackage[utf8]{inputenc}

\title{Lista de algoritmos aleatorizados}
\author{Gustavo Higuchi}
\date{\today}

\usepackage{natbib}
\usepackage{graphicx}
\usepackage{amssymb}
\usepackage{amsthm}
\usepackage{amsmath}
\usepackage{color}   %May be necessary if you want to color links
\usepackage[portuguese, ruled, linesnumbered]{algorithm2e}
\usepackage[normalem]{ulem}

\usepackage{mathtools}
\DeclarePairedDelimiter\ceil{\lceil}{\rceil}
\DeclarePairedDelimiter\floor{\lfloor}{\rfloor}

% usado para linkar cada section na tabela de conteúdo com a respectiva
% página no documento
\usepackage{hyperref}
\hypersetup{
    colorlinks,
    citecolor=black,
    filecolor=black,
    linkcolor=black,
    urlcolor=black,
    linktoc=all
}

%o começo do documento
\begin{document}

% compila o título
\maketitle

% compila a tabela de conteúdos
\tableofcontents
\newpage


\chapter{}
\section{Exercício 1}


\begin{itemize}
    \item Find $c_1$
    Para achar a probabilidade de se ter no máximo 1 em cada cesto
    \begin{equation}
    \notag
        \begin{split}
        Pr(maximo=1) &= (1-\frac{1}{n})(1-\frac{2}{n})\cdots (1-\frac{c_1\sqrt{n}}{n}) \\
        &\le \prod_{i=1}^{c_1\sqrt{n}-1} \left(1 - \dfrac{i}{n}\right) \\
        &\le \prod_{i=1}^{c_1\sqrt{n}-1} e^{-\frac{i}{n}} \\
        &= e^{\frac{c_1\sqrt{n}-c_1^2n}{2n}} \\
        &\le e^{\frac{c_1-c_1^2}{2}}
        \end{split}
    \end{equation}
    Setando $\frac{c_1-c_1^2}{2} \le -1$, obtemos $c_1 \ge 2$.
    \item
    Para a segunda parte, temos que a chance de não ter dois em um mesmo cesto é de pelo 
    menos
    \begin{equation}
    \notag
        \begin{split}
        Pr(MaxLoad=1) &= (1-\frac{1}{n})(1-\frac{2}{n})\cdots (1-\frac{c_2\sqrt{n}}{n}) \\
        &\ge \prod_{i=1}^{c_2\sqrt{n}-1} e^{-\frac{i}{n}-\frac{i^2}{n^2}} \\
        &= e^{ -\frac{c_2\sqrt{n}(c_2\sqrt{n})}{2n} -\frac{(c_2\sqrt{n}-1)c_2\sqrt{n}(2c_2\sqrt{n}-1)}{6n^2}} \\
        &\ge e^{ -\frac{c_2^2}{2} -\frac{2c_2^3n\sqrt{n}}{6n^2} } \\
        &\ge e^{ -\frac{c_2^2}{2} -\frac{c_2^3}{3} } \\
        &\ge 1 -\frac{c_2^2}{2} -\frac{c_2^3}{3}
        \end{split}
    \end{equation}
    Setando $1 -\frac{c_2^2}{2} -\frac{c_2^3}{3} \ge \frac{1}{2}$, obtemos $0 \le c_2 \le 0.8$ .
\end{itemize}

\section{Exercício 2}
\end{document}
